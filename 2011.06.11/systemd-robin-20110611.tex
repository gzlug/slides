\documentclass[CJK,xetex]{beamer}
\usepackage{xltxtra,fontspec,xunicode}
\usepackage[slantfont,boldfont]{xeCJK}
%\usepackage[usenames,dvipsnames]{color}
%\usepackage{verbatim}
%\setCJKmainfont{AR PL UMing CN}
\setCJKmainfont{WenQuanYi Zen Hei}
%\setCJKmainfont{WenQuanYi Micro Hei}
%\setCJKmainfont{AR PL KaitiM GB}
\punctstyle{kaiming}
%\setmonofont{DejaVu Sans Mono}
%\setmainfont{WenQuanYi Zen Hei}

\usepackage{tikz}
\usetikzlibrary{arrows,backgrounds,shapes.arrows,decorations.pathmorphing,fit,chains,positioning}

\mode<presentation>
{
  % \usetheme{Warsaw}
  \usetheme{PaloAlto}
  % or ...

  \setbeamercovered{transparent}
  % or whatever (possibly just delete it)
}

%\usepackage{beamerthemesplit}
%\includeonlyframes{this}
\newcommand{\tikzemph}[1]{\normalsize\textcolor{green!50!black}{#1}}
\newcommand{\mynode}[2]{node[inner sep=1pt,anchor=north,rectangle split,rectangle split parts=2] {#1 \nodepart{second} #2}}
\newcommand{\myannode}[2]{node[text width=30pt,inner sep=1pt,anchor=north,rectangle split,rectangle split parts=2] {#1 \nodepart{second} #2}}
\newcommand{\mynwnode}[2]{node[inner sep=1pt,anchor=north west,rectangle split,rectangle split parts=2] {#1 \nodepart{second} #2}}
\newcommand{\mynenode}[2]{node[inner sep=1pt,anchor=north east,rectangle
  split,rectangle split parts=2] {#1 \nodepart{second} #2}}
\newcommand{\mydir}[1]{{\color{red} \texttt{#1}}}
\newcommand{\myfile}[1]{{\color{purple} \texttt{#1}}}
\title{systemd简介}
\author[李瑞彬]{李瑞彬\\Robin `cheese' Lee\\
(cheeselee@fedoraproject.org)}
\date{2011年6月11日}


%\AtBeginSubsection[]
%{
  %\begin{frame}<beamer>{Outline}
    %\tableofcontents[sectionstyle=show/hide,subsectionstyle=show/shaded/hide] %currentsection,currentsubsection,hideallsubsections]
  %\end{frame}
%}
%\AtBeginSection[]
%{
  %\begin{frame}<beamer>{Outline}
    %\tableofcontents[currentsection,hideothersubsections]
  %\end{frame}
%}
%\newtheorem{mydef}{定义}
%\newtheorem{definition}{

\begin{document}

%\frame[plain]{\titlepage}
\frame{\titlepage}

%\begin{frame}[plain]{Outline}
%  \tableofcontents[hideallsubsections]
  % You might wish to add the option [pausesections]
%\end{frame}

\section{缘起}
\subsection{init 守护进程}
\begin{frame}{init 守护进程}
\begin{itemize}[<+->]
  \item 系统的第一个进程,由内核启动,进程号(PID)为1
% 对于所有基于Unix的系统,有那么一个程序,它由最先被启动
  \item \myfile{/sbin/init}
  \item init => initialization
  \item 进程树的根
  \item 开机,也就是初始化用户环境(bring up \emph{userspace})
  \item 关机,重启,管理系统服务……
% 它是所有(用户空间)进程的父进程
\end{itemize}
\end{frame}

\subsection{SysVInit}
\begin{frame}{System V Init}
  \begin{itemize}[<+->]
  \item 初始化脚本 (initscript / rc script)
  \item rc => run commands
  \item \mydir{/etc/rc.d/init.d}
  \item \mydir{/etc/inittab}
    \begin{itemize}
    \item \texttt{id:5:initdefault:}
    \end{itemize}
  \item \mydir{/etc/rc.d/rc5.d}
    \begin{itemize}
  \item 举例:\\
    \myfile{/etc/rc.d/rc5.d/K36mysqld}
    \begin{tabular}[]{r|c|l}
      \hline
      K & 36 & mysqld \\
      K[ill]/S[tart] & 启动顺序 & 服务名\\
      \hline
    \end{tabular}
  \item 链接至\\
    \myfile{../init.d/mysqld}\\
  \end{itemize}
\item 运行级配置相互独立
  \end{itemize}
\end{frame}

\subsection{Upstart}
\begin{frame}{Upstart}
\begin{itemize}[<+->]
\item 事件驱动(event-driven/-based)
\item 理顺服务依赖关系以便实现并行启动
\item 按需启动
\item Upstart job script
\end{itemize}
\end{frame}

\subsection{systemd}
% 照片上侧
\begin{frame}{systemd}
  \centerline{\XeTeXpicfile lennart.png width 2.5cm}
  \centerline{Lennart Poettering}
  \begin{itemize}[<+->]
  \item Red Hat 员工
  \item 但 systemd 不是 Red Hat 项目
  \item freedesktop.org 项目
  \item \url{http://www.freedesktop.org/wiki/Software/systemd}
  \end{itemize}
\end{frame}

\section{野望}
\subsection{三个中心思想}
\begin{frame}{systemd的三个中心思想}
  \begin{itemize}[<+->]
  \item 按需启动
    \begin{itemize}
    \item 启动时
    \item Socket
    \item D-Bus
    \item 时钟,可替代crond
    \item \ldots
    \end{itemize}
  \item 无竞争的并行启动(race-free parallelized start-up)
% message queue
% make those sockets available for connection earlier and only actually wait for that instead of the full daemon start-up
  \item 设定进程运行环境(context)
% 办公室上班
    \begin{itemize}
    \item CPU/IO 优先级
    \item 文件目录读写权限
    \item control group (cgroup)
% keep track of processes, every process gets its own cgroup (control group)
    \item 文件句柄(file handles)
    \end{itemize}
  \end{itemize}
\end{frame}

\section{实现}
% 刚好就是 Upstart 所没有的
\subsection{systemd units}
\begin{frame}{systemd units}
  \begin{itemize}[<+->]
  \item 一个配置文件即是一个systemd unit
  \item systemd 系统级配置目录
    \begin{itemize}
    \item \mydir{/etc/systemd/system/}
    \item \mydir{/lib/systemd/system/}
    \end{itemize}
  \item 兼容LSB initscript的头部信息
    \begin{itemize}
    \item LSB initscript 成了systemd的另一种配置方式
    \item 以systemd的方式启动服务,并不会直接运行 initscript
    \end{itemize}
  \item systemd unit的类型举例\\
    \begin{tabular}[]{l|l}
      \hline
      \texttt{.service} & 系统服务\\
      \texttt{.target}  & 运行级(runlevel)\\
      \texttt{.timer}   & 时钟\\
      \texttt{.socket}  & IPC/Network socket\\ 
      \hline
    \end{tabular}
  \end{itemize}
\end{frame}

\begin{frame}{systemd units读取示例}
  \begin{itemize}[<+->]
  \item 寻找: \myfile{default.target}
  \item 找到: \myfile{/etc/systemd/system/default.target}
  \item 链接至: \myfile{/lib/systemd/system/runlevel5.target}
    \begin{itemize}
    \item 默认运行级为5
    \end{itemize}
  \item 链接至: \myfile{/lib/systemd/system/graphical.target}
    \begin{itemize}
    \item 运行级5为图形界面环境
    \item 终于找到一个普通文件,读取它吧!
% [打开文件]
    \item \texttt{Requires=multi-user.target}
      \begin{itemize}
      \item 需要多用户环境
      \end{itemize}
    \end{itemize}
  \item 图形界面环境还需要哪些服务: \mydir{graphical.target.wants}
    \begin{itemize}
    \item \myfile{/etc/systemd/system/graphical.target.wants/*}
    \item \myfile{/lib/systemd/system/graphical.target.wants/*}
    \item \myfile{/lib/systemd/system/display-manager.service}
      \begin{itemize}
% [打开文件]
      \item \texttt{ExecStart=/etc/X11/prefdm -nodaemon}
      \end{itemize}
    \end{itemize}
  \end{itemize}
\end{frame}

\subsection{Buffered interfaces}
\begin{frame}{Buffered interfaces}
  \begin{itemize}[<+->]
  \item Socket-activatable
  \begin{itemize}
  \item 进程间通信,通过一些界面 interface 进行
  \item 最常见的就是 Unix domain socket
  \item 一个进程实际上并不依赖另一个进程,而只依赖 socket
  \item systemd 利用内核缓存机制,先产生需要的 socket
  \item 系统服务就可以无分先后地启动,而不会丢失信息
  \end{itemize}
  \item Bus-activatable
% upstart 也有
  \item Path-activatable
  \item \ldots
  \end{itemize}
\end{frame}

\begin{frame}[label=this]{Socket-activatable}
  \begin{tikzpicture}[start chain,
    node distance=5mm,
    terminal/.style={rectangle,minimum size=6mm,very
      thick,draw=red!50!black!50,top color=white,bottom
      color=red!50!black!20},
    nonterminal/.style={circle,minimum size=6mm,very
      thick,draw=red!50!black!50,top color=white,bottom
      color=red!50!black!20},
    every node/.style={}, every join/.style={->}]
    \uncover<1->{\node (systemd) [terminal] {systemd};}
    \uncover<2->{\node (socket1) [nonterminal,right=of systemd] {socket};}
    \uncover<3->{
    \node (reader1) [terminal,below right=of socket1] {Reader};
    \node (writer1) [terminal,below=of reader1] {Writer1};
    \node (writer2) [terminal,below=of writer1] {Writer2};
  }
  \uncover<4->{
    \node (write1)  [nonterminal,right=of writer1] {写};
    \node (write2)  [nonterminal,right=of write1] {写};
    \node (write3)  [nonterminal,right=of writer2,xshift=13mm] {写};
  }
  \uncover<7->{\node (write4)  [nonterminal,right=of write3,xshift=5mm] {写};}
    \uncover<5->{
      \node (finish)  [terminal,right=of reader1,xshift=8mm] {Reader};
    }
    \uncover<6->{
    \node (socket2) [nonterminal,right=of finish] {socket};
    \node (read) [nonterminal,right=of socket2,xshift=-5mm] {读};
    }

    \uncover<2->{\path (systemd) edge [->] (socket1);}
    \uncover<3->{
    \path (socket1.south) edge [->] (reader1.west);
    \path (socket1.south) edge [->] (writer1.west);
    \path (socket1.south) edge [->] (writer2.west);
  }
 \uncover<5->{\path (reader1) edge [->] (finish);}
    \uncover<4->{
    \path (writer1) edge [->] (write1);
    \path (write1) edge [->] (write2);
    \path (writer2) edge [->] (write3);
    }
    \uncover<7->{\path (write3) edge [->] (write4);}
    \uncover<6->{
    \path (finish) edge [->] (socket2);
    \path (socket1.east) edge
    [->,decorate,decoration={snake,amplitude=.4mm,segment length=2mm,post
      length=1mm}] (socket2.north);
    \node [gray,above] at (socket2.north) {\footnotesize{传递}};
    }
    \uncover<2->{\node [gray,above] at (socket1.north) {\footnotesize{创建}};}
    \uncover<3->{\node [gray,above] at (reader1.north) {\footnotesize{开始启动}};}
    \uncover<5->{\node [gray,above] at (finish.north) {\footnotesize{完成启动}};}
  \end{tikzpicture}
\end{frame}

\section{革新}
\subsection{对程序开发的影响}
\begin{frame}{systemd对程序开发的影响}
  \begin{itemize}[<+->]
  \item systemd-style daemon
    \begin{itemize}
    \item 实现一个 daemon => 请以 daemon 的方式把程序启动
    \item 不需要(old-style daemon):
      \begin{itemize}
      \item double fork/setsid/chdir/\ldots
      \item initscript
      \item \ldots
      \end{itemize}
    \item 需要(推荐):
      \begin{itemize}
      \item \texttt{.service} (systemd unit file)
      \item \texttt{SIGTERM} => 关闭
      \item \texttt{SIGHUP} => 重启
      \item \texttt{sd\_notify(3)}
      \item \ldots
      \end{itemize}
    \end{itemize}
  \item 批量启动和关闭程序
    \begin{itemize}
    \item 取代 desktop session (gnome-session/kdeinit)
    \end{itemize}
  \end{itemize}
\end{frame}

\section{结语}
\begin{frame}{结语}
  \begin{itemize}
  \item<1-> Appreciation for launchd from Apple
    \begin{itemize}
    \item<1-> 虽然对于 Linux 来说是革命性的,但 idea 其实很多来源于 launchd
    \item<2-> Mac OS X v10.4 ``Tiger'' 2005年
    \end{itemize}
  \item<3-> 参考:
    \begin{itemize}
    \item 作者初始story:\url{http://0pointer.de/blog/projects/systemd.html}
      % blog update update2
    \item \texttt{systemd.*(5)}
    \item \texttt{daemon(7)}
    \end{itemize}
  \end{itemize}
\end{frame}
\end{document}
